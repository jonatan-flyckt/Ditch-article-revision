
\documentclass[review]{elsarticle}
\graphicspath{ {./images/} }
\usepackage{hyperref}
\usepackage{float}
\usepackage{verbatim} %comments
\usepackage{apalike}
\restylefloat{figure}
\restylefloat{table}

\journal{Expert Systems with Applications}

\bibliographystyle{model5-names}\biboptions{authoryear}

\begin{document}
\begin{frontmatter}


\begin{titlepage}
\begin{center}
\vspace*{1cm}

\textbf{ \large Detecting ditches using supervised learning on high-resolution digital elevation models}

\vspace{1.5cm}

Jonatan Flyckt$^{a}$ (fljo1589@student.ju.se), Filip Andersson$^{a}$ (anfi1632@student.ju.se), Niklas Lavesson$^a$ (niklas.lavesson@ju.se), Liselott Nilsson$^b$ (liselott.nilsson@skogsstyrelsen.se), Anneli M. \AA gren$^c$ (anneli.agren@slu.se) \\

\hspace{10pt}

\begin{flushleft}
\small  
$^a$ J\"onk\"oping Artificial Intelligence Laboratory, J\"onk\"oping University, Gjuterigatan 5, 551 11, J\"onk\"oping, Sweden \\
$^b$ Forest Department, Swedish Forest Agency, Skeppargatan 17, 931 32 Skellefte\aa, Sweden \\
$^c$ Department of Forest Ecology and Management, Swedish University of Agricultural Sciences, SLU, Skogsmarksgr\"and 17, 901 83, Ume\aa, Sweden

\begin{comment}
Clearly indicate who will handle correspondence at all stages of refereeing and publication, also post-publication. Ensure that phone numbers (with country and area code) are provided in addition to the e-mail address and the complete postal address. Contact details must be kept up to date by the corresponding author.
\end{comment}

\vspace{1cm}
\textbf{Corresponding Author:} \\
Niklas Lavesson \\
J\"onk\"oping Artificial Intelligence Laboratory, J\"onk\"oping University, Gjuterigatan 5, 551 11, J\"onk\"oping, Sweden \\
Tel: (+46) 70-5383338 \textbf{Ska detta nummer stå här? (corresponding author måste ha telefonnummer)} \\
Email: niklas.lavesson@ju.se

\end{flushleft}        
\end{center}
\end{titlepage}

\title{Detecting ditches using supervised learning on high-resolution digital elevation models}

\author[ju]{Jonatan Flyckt\corref{equal_contribution}}
\ead{fljo1589@student.ju.se}

\author[ju]{Filip Andersson\corref{equal_contribution}}
\ead{anfi1632@student.ju.se}

\author[ju]{Niklas Lavesson \corref{cor1}}
\ead{niklas.lavesson@ju.se}

\author[fa]{Liselott Nilsson}
\ead{liselott.nilsson@skogsstyrelsen.se}

\author[slu]{Anneli M. \AA gren}
\ead{anneli.agren@slu.se}

\cortext[cor1]{Corresponding author.}
\address[ju]{J\"onk\"oping Artificial Intelligence Laboratory, J\"onk\"oping University, Gjuterigatan 5, 551 11, J\"onk\"oping, Sweden}
\address[fa]{Forest Department, Swedish Forest Agency, Skeppargatan 17, 931 32 Skellefte\aa, Sweden}
\address[slu]{Department of Forest Ecology and Management, Swedish University of Agricultural Sciences, SLU, Skogsmarksgr\"and 17, 901 83, Ume\aa, Sweden}
\cortext[equal_contribution]{Jonatan Flyckt and Filip Andersson contributed equally to this work as first authors.}

\begin{abstract}
In this study, we develop a method for detecting ditches in high resolution digital elevation models derived from LiDAR scans. Thresholding methods using digital terrain indices, such as Sky View Factor, Impoundment Index, or High Pass Median Filter can be used to detect ditches. However, a single threshold generally does not capture the variability in the landscape and generates many false positives and negatives. We hypothesise that, by combining the digital terrain indices using machine learning, we can improve ditch detection at a landscape-scale. In addition to the raw terrain indices, additional input variables are generated by transforming the data by including neighbouring cells to improve the predictability of the ditches. Random Forests is used to locate the ditches, and the probability output from this classifier is processed to remove noise, and binarised to produce the final ditch prediction. The confidence interval for the Cohen's Kappa index ranges \textbf{[0.655 , 0.781]} between the evaluation plots with a confidence level of \textbf{95\%}. Input variables based on the Impoundment Index overall represents the strongest ditch predictors. The study demonstrates that combining information from  a suite of digital terrain indices using machine learning provides an effective technique for automatic ditch detection at a landscape-scale.
\end{abstract}

\begin{keyword}
Machine learning \sep Geographic information systems \sep Classification and regression trees \sep Supervised learning by classification
\end{keyword}

\end{frontmatter}

\section{Introduction}
\label{introduction}



\section*{Acknowledgements}



\bibliography{references}

\end{document}
